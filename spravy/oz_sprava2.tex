% !TeX spellcheck = sk_SK

%%%%%%%%%%%%%%%%%%%%%%% file typeinst.tex %%%%%%%%%%%%%%%%%%%%%%%%%
%
% This is the LaTeX source for the instructions to authors using
% the LaTeX document class 'llncs.cls' for contributions to
% the Lecture Notes in Computer Sciences series.
% http://www.springer.com/lncs       Springer Heidelberg 2006/05/04
%
% It may be used as a template for your own input - copy it
% to a new file with a new name and use it as the basis
% for your article.
%
% NB: the document class 'llncs' has its own and detailed documentation, see
% ftp://ftp.springer.de/data/pubftp/pub/tex/latex/llncs/latex2e/llncsdoc.pdf
%
%%%%%%%%%%%%%%%%%%%%%%%%%%%%%%%%%%%%%%%%%%%%%%%%%%%%%%%%%%%%%%%%%%%


\documentclass[runningheads,a4paper]{llncs}

\usepackage{amssymb}
\setcounter{tocdepth}{3}
\usepackage{graphicx}
\usepackage{epstopdf}
\usepackage[slovak]{babel}
\usepackage[utf8]{inputenc}
\usepackage[bottom]{footmisc}

\usepackage{url}
\begin{document}
\title{Predikcia popularity článkov}
\subtitle{Objavovanie znalostí}
\titlerunning{Predikcia popularity článkov}
\author{Martin Číž, Peter Uherek}
\institute{Fakulta informatiky a informačných technológií,\\
Slovenská technická univerzita}
\authorrunning{Predikcia popularity článkov}
\maketitle


\section{Úvod}
V súčasnosti sa na webe takmer všetko podriaďuje analytike, čítanosti a 
návštev-nosti s cieľom udržať na svojom webovom sídle čo najviac používateľov na 
čo najdlhší čas. Prevádzkovatelia portálov, ktoré prinášajú rozsiahly a 
dynamický informačný obsah (typicky internetové vydania novín, týždenníkov) 
disponujú veľkým množstvom sekvenčných dát, ktoré zachytávajú správanie sa 
používateľov, históriu ich postupného prechádzania danej webovej lokality. 
Taktiež uchovávajú informácie o svojich článkoch, či už je to samotný názov, 
obsah, dátum vydania alebo téma, ktorej sa článok venuje. Z týchto dát je možné 
nielen spätné vyhodnocovanie čítanosti jednotlivých článkov, ale napr. aj 
predpovedanie popularity jednotlivých článkov či tém, čo môže následne ovplyvniť 
rozloženie webovej stránky. 

Našou úlohou bude zostrojiť model predikcie popularity článkov na základe týchto dát. Popularita článkov je z pohľadu času nestála. Inú popularitu môže mať článok hodinu od vydania a inú deň po vydaní. V našom kontexte budeme popularitu chápať ako počet všetkých prístupov k článku po dobu jedného dňa od jeho vydania. Pre predikovanie počtu prístupov použijeme niektorú z metód regresie v strojovom učení. Zaujímavý vplyv na popularitu článku môže mať téma článku, názov článku alebo autor článku preto by sme sa radi zamerali aj na skúmanie vplyvu týchto faktorov na samotnú popularitu článkov.

\section{Opis dát}
V rámci diplomovej práce sme od vedúceho obdržali z časti predspracované dáta z 15-dňového dumpu internetového vydania denníka SME. Pôvodný dump obsahoval iba jednu tabuľku, v ktorej sa nachádzali údaje prístupov používateľov na články stránky sme.sk. Tieto dáta boli rozšírene v diplomovej práci Pavla Sopka \cite{diplomovka} o dalších 5 tabuliek, ktoré obsahujú rozšírené dáta o článkoch. Pre náše potreby sme tieto rozšírené dáta využili, čo nám uľahčilo fázu predspracovania. Najzaujimavejšie z týchto dát sú dve tabuľky a to tabuľka visits a tabuľka articles.

Tabuľka visits obsahuje vyčistené dáta pôvodného dumpu a je tvorená nasledovnými stĺpcami:
\begin{itemize}
\renewcommand{\labelitemi}{$\bullet$}
  \item Časová pečiatka vytvorenia záznamu
  \item Cookie použviateľa, poďla, ktorej budeme identifikovať jedinečnosť používateľa
  \item IP adresa používateľa
  \item ID článku, na ktorý používateľ pristupoval
  \item ID miesta alebo článku odkiaľ používateľ pristupoval
  \item Dostupné informácie o prístupe používateľa (počítač, mobil, tablet)
  \item Lokácia odkiaľ používateľ pristupoval (mesto, štát)
\end{itemize}

Tabuľka articles je tvorená informáciami o článkoch a má nasledovné stĺpce:
\begin{itemize}
\renewcommand{\labelitemi}{$\bullet$}
  \item ID sekcie a ID kategórie
  \item Titulok článku
  \item Obsah článku s html značkami
  \item Obsah článku bez html značiek (čistý text) 
  \item Lematizovaný titulok článku
  \item Lematizovaný obsah článku
  \item Hodnota všetkých lematizovaných slov z metódy TFI-DF 
  \item Počet všetkých návštev (nie unikátných)  
  \item Dátum publikovania
\end{itemize}

\section{Práce iných autorov}
Naším prvým zdrojom je diplomová práca od Pavla Sopka \cite{diplomovka}. 
V práci sme sa bližšie zoznámili s postupom spracovania slovenského textu ako je odstraňovanie stop slov, odstránenie interpunkčných znamienok, lematizácia, či hľadanie prídav-ných mien medzi slovami. 

Predpoveďou popularity podobne ako my sa zaoberá aj článok  \cite{pulse}. Základne otázky, ktoré tento článok rozoberá sú:
\begin{itemize}
\renewcommand{\labelitemi}{$\bullet$}
  \item Ovplyvňuje kategória článku jeho populárnosť?
  \item Uprednostňujú čitatelia fakty, alebo preferujú citovo zararbený text?
  \item Je v tom nejaký rozdiel ak článok spomenie slávnu osobu? Záleží to od autora článku?
\end{itemize}

Práca skúma predpovedanie popularity článkov ešte pred ich vydaním. Vytvorí sa viacrozmerný priestor na základe viacerých vlastností článku a odhaduje sa úspešnosť týchto vlastností, na základe ktorých sa určuje populárnosť. Práca skúma regresné aj klasifikačné metódy strojového učenia.

Článok \cite{topic} sa venuje výberom populárnych článkov na titulnú stranu, pričom používa modelovanie podľa témy (Topic Modeling). Práca konštatuje, že odporúčanie dôležitých novinových článkov z veľkej sady je v prípade použitia len obsahu článkov ťažkou úlohou.

\section{Postup}
\subsection{Predspracovanie}
Veľká časť z dát už bola predspracovaná, ale i napriek tomu niektoré stĺpce neobsahovali požadované hodnoty a preto ich bolo potrebné dodatočne dopracovať. Jedným z takýchto stĺpcov bol stĺpec lematizovanej titulky a obsahu článku v tabuľke articles. Z celkového počtu 151.577 článkov bolo lematizovaných iba 49.300 (33 \%) článkov a nelematizovaných 102.277 (67 \%) článkov. Pre naše potreby nám tento počet pôvodne lematizovaných článkov nestačí, keďže väčšina týchto článkov nebola vydaná v čase zbierania datasetu a z toho dôvodu ich nemôžeme použiť. Preto bolo potrebné nájsť spôsob ako lematizovať ostatné články.
 
Pre lematizáciu článkov sme napísali skript, ktorý každý obsah článku upravil v nasledovnej postupnosti:
\newline
\newline
text bez html značiek $\rightarrow$ text bez interpunkčných znakov $\rightarrow$ text bez stop slov $\rightarrow$ lematizovaný text $\rightarrow$ lematizovaný text bez stop slov
\newline

Pred lematizáciou odstraňujeme stop slová, aby sme do lematizátora podávali čo najmenší počet slov. Na samotnú lematizáciu článkov sme použili voľne dostupnú web službu našej fakulty - lemmatizer\footnote{dostupný na http://text.fiit.stuba.sk/lemmatizer/index.html (2015)}. Lematizér sme použili s nastavením rýchlej lematizácie, kde výstupom je plain-text obsahujúci najpravdepodobnejšie lemy oddelené medzerami.

Po lematizácií opätovne odstraňujeme stop slová, ktoré neboli zachytené pred lematizáciou a po lematizácií môžu byť nájdene v zozname stop slov.
Ako zoznam stop slov sme použili voľne dostupný projekt na google\footnote{Dostupné na https://code.google.com/p/stop-words/source/browse/trunk/stop-words/stop-words/stop-words-slovak.txt?spec=svn3\&r=3 (2015)}, ktorý však bolo nutné ručne doplniť o desiatky slov. Procesom lematizácie neprešlo 1555 článkov z dôvodu neznámeho kódovania na strane databázy. V nasledujúcich riadkoch je uvedený príklad lematizovaného textu: 

\begin{multicols}{2}
Pred: Minulý rok si študenti Gymnázia na Grösslingovej ulici založili Gamčácke divadlo ochotníkov. V krátkom čase debutovali komédiou MAThRIX a vďaka jej úspechu sa divadlo rozhodlo pripraviť ďalší projekt. Predstavili ho už v uplynulú sobotu v DK Lúky premiéru mala ich nová komédia Taká obyčajná mafia.

Po: minulý rok študent gymnázium grösslingovej ulica založiť gamčácke divadlo ochotník krátky čase debutovať komédia mathrix vďaka úspech divadlo rozhodnúť pripraviť ďalší projekt predstaviť uplynulý sobota dk lúka premiéra mala nový komédia obyčajná mafia
\end{multicols}

\subsection{Transformácia}
Takto lematizovaný text budeme následne spracovávať pomocou metódy TD-IDF, ktorá pre každý článok vyrobí špecifický vektor, ktorého hodnoty predstavujú ako často sa slová nachádzajú v danom článku. Takto vyrobený vektor môžeme použiť vo viacerých metódach strojového učenia.

\subsection{Predpokladaný scenár použitia}
Ako bolo spomínané v úvode, popularitu článku nemôžme empiricky zhodnotiť bez určeného časového obdobia (po jednej hodine sa článok nemusí ešte uchytiť, po jednom dni už môže článok stagnovať na popularite). Väčšina článkov nadobúda po zverejnení kopček (tzv. peak) maximálnych čítaní v krátkom čase, po ktorom začína čítanosť rýchlo stagnovať a zvyšuje sa už len o priemerné hodnoty. Na základe toho sme definovali popularitu ako počet čítaní po publikovaní v krátkom časovom okne, napríklad po 10 hodinách. Našim cieľom je pokryť toto maximálne dosiahnutie čítanosti v definovanom krátkom čase. Rozhodli sme sa brať do úvahy iba čas po vydaní, ignorujeme možné periodické výkyvy popularity (napríklad članok o jesennej móde bude pravidelne čítaný na jeseň), pretože nás zaujímajú iba články, ktoré nadobudli svoju popularity po ich publikovaní.

Popularitu článkov budeme predikovať pomocou metódy viacnásobnej lineár-nej regresie, ktorá skúma vzťahy medzi viacerými premennými. V našom prípade premenné použité vo viacnásobnej lineárnej regresie budú vektory z metódy TD-IDF a jednotlivé prístupy k článkom. V prípade, ak by viacnásobna lineárna regresia nedosahovala požadované výsledky vektor prevedieme na logaritmický tvar a použijeme ho opätovne vo viacnásobnej lineárnej regresií.

Problémom pri takomto prístupe môžu byť outliari, teda články, ktoré sú počas dňa najviac čítane a svojou popularitou vyčnievajú od zvyšku článkov. Približne sa každý deň vydajú 3 takéto články z celkového počtu 400 článkov za jedeň deň. Riešením tohto problému bude odstranenie takýchto článkov z našej množiny dát.  

\subsection{Vyhodnocovanie}
Na vyhodnotenie metódy budeme používať krížovú validáciu. Dáta rozdelíme na trénovaciu a testovaciu množinu a poďla výsledkov testovacej množiny vyhodnotíme presnosť našej metódy. Tento proces budeme opakovať vždy s inou trénovaciou a testovaciou množinou. 

Presnosť výsledkov popularity budeme merať podľa premennej r2, ktorá môže nadobúdať hodnoty v intervale 0 až 1, kde hodnota nula znamená, že dáta nesedia pre danú krivku lineárnej regresie a hodnota 1 znamená, že dáta presne kopírujú krivku lineárnej regresie.

\begin{thebibliography}{7}
\bibitem{diplomovka} Sopko, Pavol, Odporúčanie novinových článkov zohľadňujúce externý kontext používateľa, 2014, FIIT STU Bratislava
  \bibitem{pulse} Bandari, Roja; Asur, Sitaram; Huberman, Bernardo A., The Pulse of News in Social Media : Forecasting Popularity, ICWSM 2012, AAAI Press
  \bibitem{topic} Toraman, Cagri, News Selection with Topic Modeling, Fifth BCS-IRSG Symposium on Future Directions in Information Access (FDIA 2013)
\end{thebibliography}

\end{document}
